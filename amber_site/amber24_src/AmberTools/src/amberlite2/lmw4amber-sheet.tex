\documentclass[10pt,landscape,twocolumn]{article}
\usepackage{multicol}
\usepackage{calc}
\usepackage{ifthen}
\usepackage{geometry}
\geometry{a4paper,margin=0.5in}
\usepackage{amsmath,amsthm,amsfonts,amssymb}
\usepackage{color,graphicx,overpic}
\usepackage{hyperref}
\hypersetup{
    colorlinks=true,
    linkcolor=blue,
    filecolor=magenta,      
    urlcolor=cyan,
}
\usepackage{mdframed}
\usepackage{multicol}
\setlength{\columnsep}{.5cm}
\setlength{\columnseprule}{0.5pt}

% Turn off header and footer
\pagestyle{empty}
\begin{document}
\section{Purpose of \textbf{\textsl{lmw4amber}}}
\textsl{lmw4amber} is used to generate Amber-related files for low-molecular-weight compounds. It was mainly created to generate files to be used in cases where more than one ligand is complexed in a protein (e.g. co-factors, ATP, and alike). It also creates a "library file" allowing to use the low-molecular-weight compound directly in PDB files. It requires AmberTools17 or later and will probably \textbf{not} work with earlier versions. 

\begin{mdframed}
\textbf{NOTE: The generated files do not include explicit solvation! For running Amber simulations on low-molecular-weight compounds in a water box, use the \textsl{files4amber} tool and read in the molecule with the \texttt{--lig} option, without specifying protein or complex. This will yield a TIP3P-water solvated system that can be used directly for explicit-water simulation.}
\end{mdframed}
\section{File types required or generated by \textsl{files4amber}}

\begin{description}
\item[*.sdf:] \textbf{input} standard SDF (MDL) file, with all hydrogens and bond orders correctly included; 
\item[*.leap.pdb:] \textbf{output} standard PDB file; the ".leap" indicates that it was created via \textsl{tleap}; the file has all hydrogens attached;
\item[*.leap.crd:] \textbf{output} Amber format coordinate file;
\item[*.leap.prm:] \textbf{output} Amber parameter-topology file;
\item[*.lib:] \textbf{output} Amber "library" file; useful for special purposes (see \ref{libfile});
\item[*.frcmod:] \textbf{output} file that contains force field parameters (generated automatically) that are not part of the original parameter files;
\item[*.ac.mol2:] \textbf{output} generated by \textsl{lmw4amber} (internally via the \textsl{antechamber} module); it has SYBYL mol2 format, but atoms have GAFF/GAFF2 force field atom types; it also contains the partial charges generated via the AM1/BCC method;
\item[sqm.*:] \textbf{output} left overs from the \textsl{sqm} module used to compute partial charges; useful for debugging only;
\item[*.leap.cmd:] \textbf{output} file containing the commands submitted to the \textsl{tleap} module; useful for debugging only;
\end{description}

\newpage
\section{Running \textsl{lmw4amber}}

For help, just type lmw4amber and RETURN. You then get this:

\begin{verbatim}
--------------------------------------------
 lmw4amber version 0.2
 Romain M. Wolf (February 2019)
--------------------------------------------
Usage: lmw4amber [options]

Options:
  -h, --help      show this help message and exit
  --lmw=FILE      ligand SD file (sdf)           (no default)
  --name=STRING   3-letter code for lmw          (default = XYZ)
  --lfrc=STRING   ligand force field             (default: gaff2)
  --chrg=INTEGER  formal charge on ligand        (default: 0)
  --rad=STRING    radius type for PB/GB          (default: mbondi2)
\end{verbatim}

\section{Command line options}
\begin{description}
\itemsep-.2em 

\item[\textbf{\texttt{--lmw}}]  must be followed by an SDF file, including all hydrogens and reflecting the correct bond order, tautomeric and protonation state; if intended to be directly integrated into a complex, the coordinates must correspond to the precise location, orientation, and conformation in that complex (\textsl{lmw4amber} does not "dock");

\item[\textbf{\texttt{--name}}] must be followed by a \textbf{3-letter} "residue" name for the low-molecular-weight compound (no extension); this feature is described in more detail below (see \ref{libfile});

\item[\textbf{\texttt{--lfrc}}] selects the (small molecule) force field; the latest GAFF (\textsl{gaff2}) is the default, the only other option would be \textsl{gaff};  

\item[\textbf{\texttt{--chrg}}] must be specified if the molecule has a formal charge; omitting this option with a charged structure leads to a failure of the partial charges computations via AM1/BCC and the routine stops;

\item[\textbf{\texttt{--rad}}] can be used to change the default selection for radii used in GB and PB computations; the \texttt{mbondi2} default is a good choice is the GB settings \texttt{igb=5} are used; keep the default if in doubt what to use;
\end{description}

\begin{quote}
\textsl{continued on the next page...}
\end{quote}
\newpage

\section{Remarks to \textsl{lmw4amber} output files}\label{libfile}
All generated files specific to the low-mol-weight compound entered under \texttt{--lmw} will start with the 3-letter name given under \texttt{--name}. 

\begin{itemize}

\item \texttt{*.leap.prm}, \texttt{*.leap.crd}), and the corresponding \texttt{*.leap.pdb}) are created by \textsl{tleap} and must be kept together; do not change the order of atoms in the files via an external application or editor;

\item \texttt{*.ac.mol2} and \texttt{*.frcmod} are the same kind of files generated also for ligands with the \textsl{files4amber} routine (see that documentation sheet); they are required when the molecule is later used as a "co-ligand" in a protein/ligand complex;

\item \texttt{*.lib} is an Amber "library" file; it can be used for more advanced features, e.g., including the \texttt{*.leap.pdb} file directly into a large PDB file (of a protein, for example), provided that the atom order, atom names, and residue name (from the \texttt{--name} option) are not altered; in that case, reading the library file via the \textsl{tleap} \texttt{loadoff} command will recognize the PDB sequence and the low-mol-weight compound is treated like a "natural" residue; the corresponding \texttt{*.frcmod} must also be loaded in \textsl{tleap} via \texttt{frcmod = loadamberparams} followed by the \texttt{frcmod} file name; the \texttt{*.lib} file type is also required to build systems with various low-mol-weight compounds, mixed solvents, etc. with programs like \textsl{packmol} (going beyond the "simple" simulations treated here...);

\end{itemize}



 
%\end{multicols}

\end{document}